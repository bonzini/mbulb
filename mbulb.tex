\documentclass{article}
\usepackage{amsmath}
\usepackage{graphicx}
\usepackage{palatino}
\usepackage{setspace}

\title{To quaternions and back}
\date{\today}
\author{Paolo Bonzini}

\let\vec\mathbf
\let\bar\overline
\newcommand{\mathnewline}[1][\quad]{\\\phantom{#1}}
\newcommand{\mathnewlineeq}[1][\quad]{\\\phantom{#1 = }}
\newcommand{\where}{\text{, where }}

\begin{document}
\maketitle

\setstretch{1.05}

\thispagestyle{empty}
\begin{abstract}
  Daniel White's \emph{Mandelbulb}, with its fractal features on all axes,
  is possibly the closest existing three-dimensional analogue to the 2D
  Mandelbrot set.

  The classic Mandelbrot set is based on the iteration of $z\leftarrow
  z^2+c$ in the complex plane, thus the most common 3D extension
  to the ``Mandelbrot'' set is the iteration of $q\leftarrow q^2+c$
  on quaternions.  The Mandelbulb, instead, extends to three dimension
  the \emph{geometric} interpretation of complex squaring.

  In this paper, I propose an alternative explanation of the Mandelbulb
  formulas using standard quaternion arithmetic, with the hope of adding
  the necessary mathematical rigor to the discovery of the Mandelbulb.
  A novel quaternion formulation is derived for the Mandelbrot iteration
  formulas; by parameterizing this formula, and appropriately choosing
  the parameters, the various Mandelbulb fractals are produced without
  resorting to non-standard algebras.
\end{abstract}

\begin{figure}[!b]
\includegraphics[width=0.45\linewidth]{mbulb2.jpg}
\hfill
\includegraphics[width=0.45\linewidth]{mbulb.jpg}

\centering
\caption{The entire exponent-8 Mandelbulb and a zoom on a 3D structure.}
\label{fig:mbulb8}
Pictures by Krzysztof Marczak and Daniel White.
\end{figure}

\clearpage

\section{Introduction}

An extension of the Mandelbrot set to three dimension is a kind of holy
grail for fractalists.  The hurdles are many and stem from the inherent
difference between the two- and three-dimensional spaces: no true analogue
of the complex field in three dimensions, no three-dimensional conformal
mappings, the \emph{hairy ball} theorem\dots\ and besides all this,
the expectations for such an object are incredibly high!

There are two main extensions of the Mandelbrot and Julia sets, both
of them to four dimensions.  One is to use quaternions instead of
complex numbers.  In order to reduce four dimensions to three, a
``stack'' of parallel 4D planes is intersected with the object and the
intersecting voxels are plotted.  The resulting objects possess
complexity, but not the infinite level of detail shown by the 2D
Mandelbrot and Julia sets---some of them are basically solids of
revolution, ``lathed'' versions of the 2D Mandelbrot set.

Another 4D expression of the Mandelbrot set does not use quaternions,
but rather composes four coordinates from the two parameters of
the iteration, $z_0$ and $c$.  Then the standard complex iteration
is used.  The 2D slices of this set are very interesting, since
they include both the Mandelbrot set (lying on the $z_0=0$ plane)
and the Julia sets (corresponding to all the constant-$c$ planes).
However, the 3D slices also fail to provide the infinite detail
and sheer beauty of their 2D correspondents.

From 2007 to 2009, Daniel White experimented with a new approach to
the quest, whose resulting fractals he named \emph{Mandelbulbs}.  In
order to understand it, I should digress shortly and explain some
properties of complex numbers.  In addition to using a purely
mathematical description in terms of complex numbers, the Mandelbrot
set's orbits can also be studied geometrically.  The real and
imaginary parts $z=a+ib$ are represented on the 2D plane by using the
real axis as the $x$ axis, and the imaginary as the $y$, and the
arithmetic operations also have a geometric interpretation.

Addition is obvious.  In order to understand multiplication, complex
numbers have to be visualized in a different way using \emph{polar}
coordinates.  The position of a point in a plane is identified by the
distance $\rho$ from the origin, and the angle $\theta$ between the
positive $x$ axis and the point.  This complex number can be written
compactly as $\rho e^{i \theta}$, and the geometric interpretation
of multiplication is derived as follows:
\begin{equation*}
  \rho e^{i \theta} =  \rho_1 e^{i \theta_1} \cdot \rho_2 e^{i \theta_2}
    = \rho_1\rho_2 e^{i (\theta_1 + \theta_2)}
\end{equation*}
  
\noindent
The result of the multiplication $z_1 z_2$ is a point that is obtained
from $z_1$ by \emph{stretching it by a factor $\rho_2$}, and
then \emph{rotating it by an angle $\theta_2$}; and the result of
squaring is obtained by squaring the distance from the origin and
doubling the angle from the positive $x$ axis.

The Mandelbulb fractal rests on the idea of extending these concepts
to three dimensions; the rest of this paper will discuss this and how
to express the result using well-known mathematical theories.
Sections~\ref{sec:2d-3d} and~\ref{sec:experimenting} explain the
discovery and math of the Mandelbulb fractal.
Section~\ref{sec:quaternions} explain the basic concepts behind
quaternions, and Section~\ref{sec:quatbulb} explain how the definition
of the Mandelbulb can be expressed using quaternions.
Section~\ref{sec:quatplex} gives quaternion definitions for other
mathematical concepts explored together with the Mandelbulb fractals.
Finally, \ref{sec:concl} concludes the paper.


\section{From 2D to 3D}
\label{sec:2d-3d}

Daniel White's intuition then was to invent a meaningful way of squaring
and adding points in 3D space.  To do the former, he had to square
the distance and somehow ``double'' the angle between the positive $x$
axis and the point in exam.  The point is then translated by $c$, and
the two steps are repeated until either the sequence diverges or a given
number of iterations is reached.  This idea had actually been proposed
before by Rudy Rucker, but his formulas had a small error that caused
them to produce less interesting results.

Many things in this scheme work out easily.  For example, Euclidean
distance in 3D space is similarly enough to 2D that the divergence
condition is the same: as soon as the distance of $z$ from the origin
exceeds 2, the iteration can be interrupted.  Other things are instead
noticeably different.  For example, it is hard to define the effect of
squaring on angles.  3D space can be described in \emph{spherical
  coordinates} that couple a distance and \emph{two}
angles---corresponding to two rotations around two different axes.
However, rotations in 3D space are \emph{not} commutative.  In fact,
applying White's idea but choosing different angles and axes gives
rise to very different fractals.

On top of this, there is some uneasiness due to the fact that 3D
complex numbers were sought by mathematicians for decades and in the
end they settled for four-component quaternions
$w+\vec{i}x+\vec{j}y+\vec{k}z$, despite these possess strange
properties such as \emph{noncommutative multiplication}.  The infinite
possible ways to define the 3D rotation give the feeling of banging
one's head against dead ends.  Searching for the properties of
three-component numbers whose multiplication is (or seems to be)
commutative, feels too much like a 21st century version of squaring
the circle.

Nevertheless, nothing suggested White's construction to be
fundamentally flawed, and it produced (very) nice pictures, especially
when the $z^2+c$ was tweaked to include higher exponents\footnote{This
  was first tried by Paul Nylander.}.  This is a winning combination
for fractalists, who proceeded to experiment with many different
definitions of rotation.  For all of them, the Mandelbulb iteration with
exponent $n$ is then obtained by the following steps:
\begin{enumerate}
\item \label{item:first-step} compute the spherical coordinates
  $(\rho,\theta,\phi)$ of the point $c=(x_c,y_c,z_c)$ being
  examined\footnote{Unfortunately, parenthesized triples may refer to
    both cartesian and spherical coordinates.  In general, the
    presence of Greek letters such as $\rho$, $\theta$, $\phi$ or
    $\pi$ will mean that a particular occurrence refers to spherical
    coordinates.};

\item \label{item:choice} compute the rotation corresponding to a
  point---remember that complex squaring, by doubling the
  polar-coordinates angle, applies a different rotation for different
  points;

\item \label{item:last-step}
  apply the rotation $n$ times to the point $(\rho^n,0,0)$;
\item translate the result by $c$.
\end{enumerate}

The urge to write this in a form that resembles Julia and Mandelbrot's
$z\leftarrow z^n+c$ is hard to resist, so White defined $z^n$ in 3D space as
applying steps \ref{item:first-step}--~\ref{item:last-step} of the
above procedure.  The actual details of exponentiation of course
depend on the actual rotation chosen for step~\ref{item:choice}.

\section{Experimenting}
\label{sec:experimenting}

White himself tried many choices, but as of today, there are three
main contestants to the title of ``best Mandelbulb rotation''.
Remember that we are defining a rotation whose details depend on the
point being rotated; hence, all three compose a rotation on the $z$ axis
and a rotation on the $y$ axis, parameterizing them by the two angles in
the spherical coordinates of a point---the \emph{azimuth} $\theta$ and
the \emph{elevation} $\phi$.

\begin{equation}
\label{eq:negz}
  R_z(\theta) \cdot R_y(\phi)
\end{equation}
\begin{equation}
\label{eq:posz}
  R_z(\theta) \cdot R_y(-\phi)
\end{equation}
\begin{equation}
\label{eq:cosine}
  R_z(\theta) \cdot R_y(\pi/2-\phi)
\end{equation}

Each of the three has interesting properties, and the first two
degenerate to $R_z(\theta)$ (and hence to the Mandelbrot set) on the
$xy$ plane.  (\ref{eq:negz}) is possibly the most natural solution and
produces a nice exponent-2 Mandelbulb, but it has the apparent
disadvantage that $(x,y,z)^1=(x,y,-z)$.  Instead, (\ref{eq:posz}) has
$(x,y,z)^1=(x,y,z)$.  Even then, (\ref{eq:cosine}) appears to be
much more weird, since $(x,y,z)^0=(0,0,1)$.

Equation~(\ref{eq:posz}) was discovered by Paul Nylander, who then proceeded to
define a fairly complete algebra with commutative (but nonassociative)
multiplication, multiplicative inverses, and division.  For example,
since
\begin{equation}
  \label{eq:posz-exp}
  (\rho,\theta,\phi)^n=(\rho^n, n\theta, n\phi)
\end{equation}

\noindent
the following definition of multiplication seems natural:
\begin{equation*}
  (\rho_1,\theta_1,\phi_1) (\rho_2,\theta_2,\phi_2)=
  (\rho_1 \rho_2,\theta_1+\theta_2,\phi_1+\phi_2)
\end{equation*}

\noindent
However, this attempt too seemed stuck against a dead end once the
proposed algebra started to show more and more annoying differences
from standard mathematical concepts.  For example, properly
calculating equation~(\ref{eq:posz-exp}) requires a definition of
$\phi$ in the range $[-\pi,\pi]$, while spherical coordinates define
elevation in the range $[-\pi/2,\pi/2]$.  This in turn means that a
cartesian representation of this algebra is not \emph{power
  associative}\footnote{In a power associative algebra, $(x,y,z)^{n+m}
  = (x,y,z)^n (x,y,z)^m$}.

Leaving aside for a moment the nice pictures, the fundamental insight
that White had is this: possibly, the essence of the 2D Mandelbrot does
not rely on the complex field---there could be something else more
fundamental to the set's appearance, and this thing could be rotations.

\section{Reconsidering quaternions}
\label{sec:quaternions}

Now, rotations are something that mathematicians have learnt to handle
very well.  They have two tools of choice to deal with rotations, namely
matrices and\dots\ quaternions.

Matrices are an extremely general tool that can represent arbitrary
linear transformations of an arbitrary vector space, including for
example those that do not preserve angles.  Any of the three equations
in the previous section could indeed be converted to a 3x3 matrix $R$,
and the resulting iteration would have this shape:
\begin{equation*}
  (\rho,\theta,\phi)^n = R(\theta,\phi)^n (\rho^n,0,0)
\end{equation*}

\noindent
Even better, the scaling operation could be included in the transformation
matrix like this:
\begin{equation*}
  (\rho,\theta,\phi)^n = R(\rho,\theta,\phi)^n (1,0,0)
\end{equation*}
\noindent
under the following conditions:
\begin{itemize}
\item $R$ affects areas by a factor of exactly $\rho$: $\det R = \rho^3$
\item $R$ is a combination of scaling and rotation: $R^{-1} = R^T / \rho^2$
\end{itemize}

If these two conditions are imposed, however, the transformation is
more compactly represented by a quaternion $q$.  \emph{Note that the
  operation that will be performed on the quaternion is \textbf{not}
  $q\leftarrow q^2+c$, so this will still result in a Mandelbulb}
rather than the disappointing quaternion Julia sets.  Only, the
3-tuples of Section~\ref{sec:2d-3d} will be replaced a well-known
mathematical object, the quaternion.

\emph{Unit quaternions}, that is quaternions whose norm is 1, represent
the space of 3D rotations in a simple way.  For the purpose of this paper,
it will suffice to show how to describe a rotation by a quaternion,
without explaining the theory behind this.  $\vec{v}$ will indicate a
quaternion with a zero scalar part, that is $\vec{ix}+\vec{j}y+\vec{k}z$;
the rotation by an angle $\alpha$ around axis $\vec{v}$ is then
represented by the following quaternion:
\begin{equation*}
q = \cos \frac\alpha2 + \vec{v} \sin \frac\alpha2
\end{equation*}

The rotation is clockwise if the observer's line of sight points in the
same direction as $\vec{v}$.  A rotation $q$ can be applied to a vector
$\vec{w}$ by computing two quaternion multiplications, specifically
$q\vec{w}q^{-1}$.

Rotations can be composed by multiplying two quaternions; ordering is
significant, since quaternion multiplication is noncommutative.  $q^n$
instead is well defined as a rotation by $n$ times the angle around the
same axis as $q$, since multiplication is still associative.

These formulas of course work in 2D too; in this case the rotated
point will be of the form $\vec{i}x+\vec{j}y$ and, in order to stay in
the $xy$ plane, the rotation must be performed around the $z$ axis:
\begin{equation*}
q = \cos \frac\alpha2 + \vec{k} \sin \frac\alpha2
\end{equation*}

\noindent
This is more compactly written $e^{\vec{k}\alpha/2}$.  The polar
representation of complex numbers also has an equivalent using
quaternions on the $xy$ plane, albeit the expression is more
complicated.  Let $\vec{v}=\vec{i}x+\vec{j}y$ be a point on the $xy$
plane, and $(\rho,\theta)$ its polar representation.  Then, it's
possible to write $\vec{v}$ from $\rho$ and $\theta$ as follows:
\begin{equation}
\label{eq:quat-v}
\vec{v}=\rho e^{\vec{k}\theta/2} \vec{i} e^{-\vec{k}\theta/2}
\end{equation}

From here it is a short step to a somewhat nontraditional formulation
of the Mandelbrot set, based on quaternion rotations.  It was already
said repeatedly that $\vec{v}^2$ corresponds to squaring $\rho$ and
doubling $\theta$.  Then, it's possible to write the Mandelbrot
iteration as follows:
\begin{equation}
  \label{eq:quat-mand-first}
\vec{v} \leftarrow \rho^2 e^{\vec{k}\theta} \vec{i} e^{-\vec{k}\theta}+\vec{c}
\end{equation}

But this complicated expression can be simplified noticeably.  For unit
quaternions, $q^{-1}=\bar q$: inversion and conjugation are the same
operation, and $q\vec{i}q^{-1}$ can also be written $q\vec{i}\bar
q$.  When the modulus is not one, instead, $q\vec{i}\bar q = |q|^2
q\vec{i}q^{-1}$.  Therefore we can simplify equation~\ref{eq:quat-mand-first}
to
\begin{equation*}
\vec{v} \leftarrow q \vec{i} \bar q+\vec{c}
\end{equation*}
  
\noindent
where
\begin{equation*}
q = \rho e^{\vec{k}\theta} = x + y\vec{k} = -\vec{v}\vec{i}
\end{equation*}

\noindent
Changing the sign of $q$ does not modify the effect of the
rotation\footnote{This is true for all quaternions, since quaternions
  are actually a \emph{double cover} of the space of rotations.}, and
it slightly simplifies the iteration.  So we can define equivalently
$q=\vec{v}\vec{i}$ and, generalizing the exponent $n$ to values other
than $2$, we get:
\begin{equation}
\label{eq:quat}
\vec{v}\leftarrow q^{n/2} \vec{i} \bar q^{n/2}+\vec{c} \where q=\vec{v}\vec{i}.
\end{equation}

\noindent
It is easy to show that putting $q=\vec{v}$ gives the same
result\footnote{Multiplying on the right by $\vec{-i}$ corresponds to
  adding a 180 degree rotation around the $x$ axis, which is
  irrelevant because the transformed point always lies on that axis.
  An interesting point in the resulting formula is that
  $\vec{v}\vec{i}\vec{\bar v}$ is also a quaternion rotation, more
  precisely a 180 degree rotation around axis $\vec{v}$.  In 2D this
  is a reflection, so the result will always lie on the $xy$ plane,
  even though the axis is not $\vec{k}$.}.  The elegant iteration that
results, $\vec{v}\leftarrow\vec{v} \vec{i} \vec{\bar v}+\vec{c}$, can
thus be seen as a \emph{quaternion formulation of the Mandelbrot set}.
However, when drawn in 3D, it results simply in a ``lathed'' 2D
Mandelbrot set.  So, for the sake of generalizing to three dimensions
(and rediscovering the Mandelbulb in its quaternion form), we'll have
to take a step back to equation~(\ref{eq:quat}), and see if it can be
actually generalized to 3D.

\section{Rethinking the Mandelbulb}
\label{sec:quatbulb}

Since there is no single candidate to the role of 3D Mandelbrot, it
makes sense to define a family of iterations that can be used to
produce these fractals.  Some of them will be more interestings and
other more boring, but all of them will share the property of
degenerating to a same-order Mandelbrot set on a plane---for
simplicity, the choice will be the $xy$ plane.

Each different fractal will be described by a function $Q(\vec{v})$
associating a quaternion to a position in 3D space $\vec{v}$.  To
satisfy the first condition, the function should satisfy
$|Q(\vec{v})| = |\vec{v}|$; the second condition can be imposed
by checking that
\begin{equation*}
\label{eq:its-mandel}
 Q(\vec{i}x+\vec{j}y) = x+\vec{k}y
\end{equation*}
\noindent
whenever the $\vec{k}$ component of $\vec{v}$ is zero (alternatively,
if $\vec{v}$'s spherical coordinates are used, whenever $\phi = 0$).

Note that this would exclude equation~(\ref{eq:cosine}), which indeed
has a very distorted appearance for $n=2$.  Still, under this definition
there are infinite possible definitions of $Q(\vec{v})$ that make sense
and many produce interesting drawings.  The main advantage is that the
function $Q(\vec{v})$ is the same for all exponents: its definition
automatically includes the generalization of squaring to exponentiation.
It is also includes possibilities such as adding phase shifts to $\phi$
or multiplying by arbitrary factors.

Let's consider for example equation~(\ref{eq:posz}) for $n=2$.  The
desired rotation angles are $2\theta$ and $2\phi$, which simplify
nicely with the fractional terms $\theta/2$ and $\phi/2$ of
equation~(\ref{eq:quat-v}). $Q(\vec{v})$ can then be expressed easily
in terms of $\vec{v}$'s components:
\begin{equation}
  \label{eq:posz-q}
  \begin{array}{l}
  Q(\vec{v}) = \rho (\cos \theta + \vec{k} \sin \theta) (\cos \phi - \vec{j} \sin \phi)
  \mathnewline[Q(\vec{v})] = \rho \cos \theta \cos \phi + \vec{i} \rho \sin \theta \sin \phi - \vec{j} \rho \cos \theta \sin \phi + \vec{k} \rho \sin \theta \cos \phi
  \mathnewline[Q(\vec{v})] = x + \vec{i}\frac{y z}{\sqrt{x^2 + y^2}} - \vec{j}\frac{x z}{\sqrt{x^2 + y^2}} + \vec{k} y
\end{array}
\end{equation}

\noindent
For $x=y=0$, the function is not continuous, and any value of $\theta$
can be used.  The simplest possibilities are $Q(\vec{v}) = -\vec{i}z$
or $Q(\vec{v}) = \vec{j}z$.

From this formula it is trivial to verify the condition expressed
above for embedding the Mandelbrot set.  It is also possible, if
needed, to easily convert the quaternion $q$ back to $(\rho, \theta,
\phi)$:
\begin{equation*}
  \rho = |q|, \theta = \arctan {-\frac{q_x}{q_y}} = \arctan {\frac{q_z}{q_w}},
  \phi = \arctan {\frac{q_x}{q_z}} = \arctan {-\frac{q_y}{q_w}}
\end{equation*}

Unfortunately, the definition of $Q(\vec{v})$ proposed so far will match the
corresponding Mandelbulb only for $n=2$.  For higher orders, raising
equations~(\ref{eq:posz-q}) and~(\ref{eq:negz-q}) to $n/2$ will
alternate rotations around the $z$ and $y$ axes, instead of making a
single rotation around the $z$ axis followed by one around the $y$
axis.

So, instead of using exponentiation as in equation~(\ref{eq:quat}),
$Q(\vec{v})$ has to be generalized a family of functions
$Q_\nu(\vec{v})$ satisfying $Q_\nu(\vec{i}x+\vec{j}y) =
(x+\vec{k}y)^\nu$.  The iteration will be defined as follows:
\begin{equation*}
\vec{v}\leftarrow q\vec{i}\bar q+\vec{c} \where q=Q_{n/2}(\vec{v})
\end{equation*}

\noindent
The family of functions corresponding to equation~(\ref{eq:posz})
becomes:
\begin{equation*}
  \begin{array}{l}
  Q_\nu(\vec{v}) = \rho (\cos \nu\theta + \vec{k} \sin \nu\theta) (\cos \nu\phi - \vec{j} \sin \nu\phi)
  \mathnewline[Q_\nu(\vec{v})] = \rho \cos \nu\theta \cos \nu\phi + \vec{i} \rho \sin \nu\theta \sin \nu\phi - \vec{j} \rho \cos \nu\theta \sin \nu\phi + \vec{k} \rho \sin \nu\theta \cos \nu\phi
  \end{array}
\end{equation*}
\noindent
This of course degenerates to equation~(\ref{eq:posz-q}) when
$\nu=1$. As another example of defining $Q_\nu(\vec{v})$,
equation~(\ref{eq:negz}) can be written as follows:
\begin{equation*}
  \label{eq:negz-q}
  \begin{array}{l}
  Q_\nu(\vec{v}) = \rho (\cos \theta + \vec{k} \sin \theta) (\cos \phi + \vec{j} \sin \phi)
  \mathnewline[Q_\nu(\vec{v})] = \rho \cos \nu\theta \cos \nu\phi - \vec{i} \rho \sin \nu\theta \sin \nu\phi + \vec{j} \rho \cos \nu\theta \sin \nu\phi + \vec{k} \rho \sin \nu\theta \cos \nu\phi
  \end{array}
\end{equation*}
 
\noindent
giving for example
\begin{equation*}
  Q_1(\vec{v}) = x - \vec{i}\frac{y z}{\sqrt{x^2 + y^2}} + \vec{j}\frac{x z}{\sqrt{x^2 + y^2}} + \vec{k} y
\end{equation*}

\section{Expressing triplex algebra using quaternions}
\label{sec:quatplex}

One important property of the values of equations~(\ref{eq:posz-q})
and following is that $Q(\vec{v})$ takes a different value for each
$\theta$ and $\phi$ in the range $(-\pi, \pi)$.  In other words,
unlike when using ``triplex numbers'', rotation of the elevation can
be expressed using its full $2\pi$ range; you may recall this to be a
problem with triplex numbers, and we can now see why this problem is
present.

The limited range of the elevation does not pose any problem in the
case of triplex exponentiation, since it groups together two steps:
conversion of from spherical coordinates to quaternions, and applying
the resulting rotation to $\vec{i}$.  In Nylander's multiplication
formula however two objects with two different meanings (a rotation
and a vector) appear in a commutative way, but since rotations are
``twice as many'' as vectors, the rotation cannot be described
exactly.  While it happens to suffice in the case of squaring, this
does not extend to higher powers.  Therefore, unlike exponentiation
and due to its non-associativity, \emph{``triplex multiplication''
  does not have a geometrical meaning} when expressed in cartesian
coordinates.  This is also related to the lack of power-associativity
in White and Nylander's 3-tuples; instead, the quaternion formulation
intrinsically avoid the problem by representing rotations and vectors
differently.

For a well behaved $Q_\nu(\vec{v})$, however, it may be possible to
define a composition operator akin to triplex multiplication while
avoiding its pitfalls.  Triplex exponentiation can be defined as
the operation that ``creates'' $Q_\nu(\vec{v})$ from $Q_1(\vec{v})$;
hence, a composition operator $\circ$ possessing the property
\begin{equation*}
  Q_\mu(\vec{v}) \circ Q_\nu(\vec{v}) = Q_{\mu+\nu}(\vec{v})
\end{equation*}

\noindent
will have all the properties of triplex multiplication, \emph{plus}
(by definition) power-associativity.  The definition of $\circ$ (if it
exists at all) of course depends on the definition of
$Q_\nu(\vec{v})$.  We are however concerned only with
equation~(\ref{eq:posz}), since this is the only case where a
power-associative multiplication operator can be defined.  In this
case, the definition of $Q_{\mu+\nu}(\vec{v})$ can be expanded as
follows:

\begin{equation*}
  \begin{array}{l}
  Q_{\mu+\nu}(\vec{v}) = \rho \cos (\mu+\nu)\theta \cos (\mu+\nu)\phi + \vec{i} \rho \sin (\mu+\nu)\theta \sin (\mu+\nu)\phi
  \mathnewlineeq[Q_{\mu+\nu}(\vec{v})] - \vec{j} \rho \cos (\mu+\nu)\theta \sin (\mu+\nu)\phi + \vec{k} \rho \sin (\mu+\nu)\theta \cos (\mu+\nu)\phi
  \mathnewline[Q_{\mu+\nu}(\vec{v})] = \rho \cos (\mu\theta+\nu\theta) \cos (\mu\phi+\nu\phi) + \vec{i} \rho \sin (\mu\theta+\nu\theta) \sin (\mu\phi+\nu\phi)
  \mathnewlineeq[Q_{\mu+\nu}(\vec{v})] - \vec{j} \rho \cos (\mu\theta+\nu\theta) \sin (\mu\phi+\nu\phi) + \vec{k} \rho \sin (\mu\theta+\nu\theta) \cos (\mu\phi+\nu\phi)
  \end{array}
\end{equation*}

\noindent
Addition formulas can be used to expand the cosines and sines in terms
of $Q_\mu(\vec{v})$ and $Q_\nu(\vec{v})$, and the result simplifies to
give the desired definition of $p \circ q$:
\begin{equation*}
\begin{array}{l}
p \circ q =
    (p_w q_w + p_x q_x - p_y q_y - p_z q_z)
  + \vec{i} (p_w q_x + p_x q_w - p_y q_z - p_z q_y) \mathnewlineeq[p \circ q]
  + \vec{j} (p_w q_y + p_x q_z + p_y q_w + p_z q_x)
  + \vec{k} (p_w q_z + p_x q_y + p_y q_x + p_z q_w)
\end{array}
\end{equation*}

This is similar to quaternion multiplication, but not exactly.  The
units of the quaternion space are composed as follows\footnote{One should
  resist the temptation to take this table as an alternative
  definition of quaternions.  This table describes simply the
  way rotations are composed in one particular kind of Mandelbulb,
  and all the other results in the paper rely on ``traditional''
  quaternion rules.  For this reason the operator is denoted by
  $\circ$.}:

\smallskip
\begin{center}
\begin{tabular}{c|cccc}
  $\circ$ & $1$ &   $i$  &  $j$  &  $k$ \\\hline
  $1$ &     $1$ &   $i$  &  $j$  &  $k$ \\
  $i$ &     $i$ &   $1$  &  $k$  &  $j$ \\
  $j$ &     $j$ &   $k$  & $-1$  & $-i$ \\
  $k$ &     $k$ &   $j$  & $-i$  & $-1$
\end{tabular}
\end{center}
\smallskip

Unlike Nylander's triplex multiplication, it is commutative \emph{and}
associative. Still, if $\ast$ denotes triplex multiplication, the
following identity also holds approximately:
\begin{equation}
\label{eq:correct-mul}
\vec{v} \ast \vec{w} \approx q\vec{i}\bar q
   \where  q=Q_1(\vec{v}) \circ Q_1(\vec{w})
\end{equation}
\noindent
The equality is not exact, because azimuths will differ by $\pi$
radians iff $Q_1(\vec{v}) \circ Q_1(\vec{w})$ includes a rotation
around the $y$ axis by more than $\pi$ radians---in this case,
however, the value of the left hand side of
equation~(\ref{eq:correct-mul}) is effectively wrong.  The
corresponding identity for exponentiation instead is exact:

\begin{equation*}
\vec{v} \ast^{n} \vec{v} = q\vec{i}\bar q
   \where  q=Q_1(\vec{v}) \circ^{n} Q_1(\vec{v})
\end{equation*}

\noindent
which is lucky because computing exponentiation directly on cartesian
coordinates is faster than going ``to quaternions and back''.
However, exact composition of arbitrary rotations requires the full
expressiveness of the $\circ$ operator or, equivalently, of a polar
representation for triplex numbers.

\section{Conclusion}
\label{sec:concl}

In this paper I proposed to describe the family of Mandelbulb fractals
using quaternions, so that each possible definition of exponentiation
can be written as a functions from a vector to a quaternion.  This
formulation has several advantages: it makes it easy to distinguish
members which embed the Mandelbrot fractal, it eliminates the need to
define new non-standard algebras, and it defines White's operators in
a way that preserves associativity and power associativity.

An alternative method using transformation matrices was also presented
briefly.  This method could be more useful for extensions beyond the
third dimension.

However, this paper barely scratched the surface in terms of analyzing
the properties of the fractals itself.  There are many questions that
can be analyzed.  For example, what are the boundaries of the set?
Or, why do some fractals have a flat appearance for $n = 2$?

Since there may be no true 3D equivalent of the Mandelbrot fractal,
having defined a space of $Q_\nu(\vec{v})$ functions may at least help
steering the search for new fractal creations of this family.  It is
my hope that this paper will provide a tool to analyze their
characteristics rigorously and, maybe, will contribute to the
discovery of new beautiful mathematical objects.

\newpage
\appendix
\section{Expansion of quaternion formulas}

\begin{description}
\item[Rotation of $\vec{i}$ by $q$]
\begin{equation*}
q\vec{i}\bar q =
  \vec{i}(q_w^2+q_x^2-q_y^2-q_z^2) + \vec{j} (2q_x q_y + 2q_w q_z)
  + \vec{k} (2 q_x q_z - 2 q_w q_y)
\end{equation*}

\noindent This is the main step in the computation of the Mandelbulb
for $n=2$.

\bigskip
\item[Quaternion squaring and multiplication]
\begin{equation*}
\begin{array}{l}
\hbox to 0pt{$q^2\hss$}\phantom{p q} =
  (q_w^2-q_x^2-q_y^2-q_z^2) + 2 \vec{i} q_w q_x + 2 \vec{j} q_w q_y
  + 2 \vec{k} q_w q_z\\
p q =       (p_w q_w - p_x q_x - p_y q_y - p_z q_z)
  + \vec{i} (p_w q_x + p_x q_w + p_y q_z - p_z q_y) \mathnewlineeq[p q]
  + \vec{j} (p_w q_y - p_x q_z + p_y q_w + p_z q_x)
  + \vec{k} (p_w q_z + p_x q_y - p_y q_x + p_z q_w)
\end{array}
\end{equation*}

\noindent
Squaring and multiplication allow to compute an arbitrary power using
the \emph{binary exponentiation} algorithm.

\bigskip
\item[``Triplex squaring'' on quaternions] ~\\
\begin{equation*}
\begin{array}{l}
p \circ p =
    (p_w^2 + p_x^2 - p_y^2 - p_z^2) + \vec{i} (2 p_w p_x - 2 p_y p_z)
\mathnewlineeq[p \circ p]
  + \vec{j} (2 p_w p_y + 2 p_x p_z) + \vec{k} (2 p_w p_z + 2 p_x p_y)
\end{array}
\end{equation*}

Using the previous identity, the binary exponentiation algorithm can
be used also to compute $Q_\nu(\vec{v})$, in particular when $\nu$ is a
power of 2.  When $\nu$ is integer the exponentiation will start
from $Q_1(\vec{v})$, i.e.~from equation~(\ref{eq:posz-q}),
otherwise from $Q_{1/2}(\vec{v})$.

%% \bigskip
%% \item[$Q(\vec{v})^2$ for the definition of equation~(\ref{eq:posz-q})]
%% \begin{equation*}
%% Q(\vec{v})^2 =
%%   (x^2-y^2-z^2) + \frac{2xz}{\sqrt{x^2+y^2}} (-\vec{i} y + \vec{j} x)
%%   + 2\vec{k}xy
%% \end{equation*}
%% \noindent This is slightly faster than computing $Q(\vec{v})$ and then squaring,
%% due to the simplifications in the $q_w$ term.  It can be used to compute
%% the Mandelbulb for $n=4$.
%% 
%% \bigskip
%% \item[Real part of $Q(\vec{v})^4$ for the definition of equation~(\ref{eq:posz-q})]
%% \begin{equation*}
%% \mathrm{Re}~Q(v)^4 = (x^2-y^2-z^2)^2 - 4x^2 y^2 -4 x^2 z^2
%% \end{equation*}
%% 
%% For $n=8$ one more squaring is necessary.  However, the real term can
%% be simplified considerably using the above expression.  Note that this
%% formula does \emph{not} give the order-8 Mandelbulb represented in
%% figure~\ref{fig:mbulb8}.

\end{description}
\end{document}
